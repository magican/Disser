\chapter*{Введение}							% Заголовок
\addcontentsline{toc}{chapter}{Введение}	% Добавляем его в оглавление


Современную научно-исследовательскую деятельность в области океанологии и метеорологии, которые плохо обеспечены контактными данными, уже невозможно представить без использования данных спутниковых наблюдений. Существующие методы обработки и анализа спутниковых измерений обеспечивают получение данных об «огромном» наборе параметров, характеризующих систему «атмосфера -- земля -- океан», которые в настоящее время широко используются в задачах мониторинга и прогноза окружающей среды. В настоящее время ряд спутниковых геофизических продуктов находится в открытом доступе. В приложении к океанографическим исследованиям, спутниковые данные могут быть получены, например, из центра данных Physical Oceanography Distributed Active Archive Center (PO.DAAC) -- американский центр НАСА (\url{http://podaac.jpl.nasa.gov/}), Centre ERS d'Archivage et de Traitement -- (CERSAT) -- французский центр данных института IFREMER (\url{http://cersat.ifremer.fr/}); информационный портал спутниковых данных РГГМУ (SATIN - \url{http://satin.rshu.ru/}).

В то же время, перспективы развития исследований Земли из Космоса неизбежно требуют создания новых подходов и методов обработки, анализа и использования спутниковой информации. Оптические методы исследования Земли являются наиболее развитыми и широко используемыми в оперативной практике. В настоящее время на орбите Земли находится большое количество сканеров, работающих в оптическом диапазоне (например, сканеры MODIS на спутниках Terra и Aqua, радиометры AVHRR на серии спутников NOAA). Одно из основных применений данных оптических сканеров, - изучение «цвета» Океана (содержание фитопланктона и минеральной взвеси, биогеохимические характеристики), а также температуры его поверхности.

При изучение оптических характеристик Океана, солнечная радиация, отраженная от морской поверхности, является шумом по отношению к радиации рассеянной в верхнем слое Океана. В областях солнечного блика отражённая радиация составляет значительную часть регистрируемого излучения, что исключает возможность применения алгоритмов восстановления «цвета» Океана. Существование солнечного блика приводит к тому, что огромная часть спутниковых сканерных данных (до 30\%) не может быть использована в классических океанографических приложениях. Области, где восстановление параметров цвета Океана по спутниковым данным невозможно, маскируется для конечного пользователя и, таким образом, «выбрасывается в мусорный ящик».

Повышение информативности данных измерений спутниковыми сканерами определяет \textbf{актуальность} данного исследования. Основная идея состоит в том, что отраженная солнечная радиация несёт информацию о характеристиках «шероховатости» поверхности Океана. В этом случае данные оптических сканеров могут быть использованы для исследования статистических характеристик ветрового волнения и их вариаций, вызванных различными океаническими процессами. Актуальность данного исследования определяется так же необходимостью разработки нового метода, позволяющего использовать отбрасываемые ранее данные оптических сканеров для исследования проявления различных динамических процессов на поверхности Океана. Предлагаемый подход, совместно с существующими радиолокационными (РЛ) методами наблюдения поверхности Океана, открывает новые возможности для мониторинга океанических явлений из Космоса по их поверхностным проявлениям.

\textbf{Основной целью работы} является разработка метода исследования поверхности Океана по спутниковым изображениям солнечного блика, и применение этого метода для исследования нефтяных загрязнений и поверхностных проявлений динамических процессов в Океане.

Для достижения поставленной цели в работе решаются следующие \textbf{задачи}:

\begin{enumerate}
\item  разработать метод восстановления пространственных вариаций среднеквадратичного наклона (СКН) морской поверхности по полю яркости солнечного блика;

\item  применить разработанный метод для анализа данных спутниковых оптических сканеров MODIS и MERIS;

\item  исследовать поверхностные проявления биологических и нефтяных сликов в солнечном блике и в поле СКН морской поверхности, а также исследовать подобие и отличия аномалий «шероховатости» морской поверхности в сликах, измеряемых оптическими и радиолокационными методами;

\item  исследовать особенности проявления внутренних волн и мезомасштабных течений на морской поверхности по изображениям солнечного блика;

\item  исследовать связь аномалий характеристик «шероховатости» морской поверхности с параметрами мезомасштабных течений на основе синергетического анализа оптических и радиолокационных изображений;

\item  разработать специализированное программно-математическое обеспечение, сопровождающее разработанные методы.
\end{enumerate}

\textbf{Научная новизна} работы состоит в следующем:

\begin{enumerate}
\item  разработан новый метод восстановления пространственных вариаций СКН морской поверхности по изображениям солнечного блика, регистрируемого спутниковыми оптическими сканерами;

\item  исследованы проявления нефтяных сликов в солнечном блике и в поле СКН морской поверхности. Показано что контрасты СКН в нефтяных сликах систематически ниже контрастов СКН в сликах биологического происхождения, эффективный коэффициент упругости тонкой нефтяной плёнки может быть задан как E=15мН/м;

\item  продемонстрировано, что наблюдения Океана в солнечном блике могут являться эффективным инструментом исследования ВВ. Поверхностные проявления ВВ видны через модуляцию среднеквадратичного наклона морской поверхности. Усиление среднеквадратичного наклона (СКН) происходит в зонах конвергенции течения ВВ, а его подавление - в зонах дивергенции;

\item  на основе совместного анализа оптических и радиолокационных изображений установлено, что мезомасштабные течения проявляются на морской поверхности в виде пространственных вариаций СКН и обрушений волн;

\item  проявления мезомасштабных течений обусловлено в основном влиянием дивергенции течений на ветровые волны. Соответственно, аномалии состояния поверхности в виде увеличения/уменьшения ее СКН и интенсивности обрушений волн привязаны к зонам конвергенции/дивергенции течений, которые в свою очередь связаны с градиентами поля завихренности квази-геострофического течения.
\end{enumerate}

\textbf{Практическая и научная значимость} работы.

Полученные научные результаты реализованы в виде алгоритмов и элементов программного обеспечения и использованы для обработки данных радиолокаторов с синтезированной апертурой (РСА) и оптических изображений, а также восстановления статистических параметров поверхности Океана.
Предложенные алгоритмы и методики были апробированы и внедрены в Международном центре по окружающей среде и дистанционному зондированию им. Нансена (NIERSC), а также в Лаборатории Спутниковой Океанографии (ЛСО, на англ. SOLab) РГГМУ, в виде элементов спутникового информационного портала SATIN (от англ. SATellite Data Search and Manage INformation Portal), для поиска, получения, отображения, распространения и хранения данных дистанционного зондирования (\url{http://satin.rshu.ru/}), а также как элемент разрабатываемой синергетической платформы SYNTool (\url{http://syntool.solab.rshu.ru/}) ЛСО РГГМУ.
В результате применения разработанных методов и алгоритмов, получена возможность использовать данные о яркости поверхности Океана внутри солнечного блика для исследования океанографических явлений по их поверхностным проявлениям, что, в свою очередь, позволило значительно расширить область применимости оптических сканеров. Показано что, применение синергетического подхода, основанного на совместном использовании РСА и оптических данных, позволяет лучше понять механизмы проявления океанических явлений на поверхности и выработать предложения по комбинации датчиков и спектральных каналов для повышения эффективности спутникового мониторинга морской среды.

\textbf{Положениями, выносимыми на защиту,} являются:

\begin{enumerate}
\item  разработанный метод диагностики пространственных аномалий «шероховатости» поверхности Океана по спутниковым изображениям солнечного блика позволяет работать с различными оптическими спектрометрами благодаря использованию передаточной функции, которая напрямую зависит от наблюдаемых градиентов яркости солнечного блика, без априорного задания плотности распределения уклонов;
\item контрасты СКН в нефтяных сликах систематически ниже контрастов СКН в сликах биологического происхождения;
\item для одного и того же слика, сформированного тонкой нефтяной плёнкой, контрасты УЭПР примерно в 1.6 раза сильнее контрастов СКН;
\item поверхностные проявления ВВ и мезомасштабных течений отчётливо проявляются в модуляциях уклонов морской поверхности в результате усиления среднеквадратичного наклона (СКН) в зонах конвергенции течения, и его подавления в зонах дивергенции;
\item аномалии характеристик ветрового волнения (СКН, обрушения) связаны с зонами дивергенции течений и пространственно привязаны к областям сильных градиентов завихренности полей квази-геострофических течений.

\end{enumerate}

\textbf{Апробация работы и публикации.}

Результаты работы докладывались на различных отечественных и международных конференциях и семинарах, в частности на 7-ой всероссийской открытой конференции «Современные проблемы Дистанционного зондирования Земли из Космоса» (Москва, Россия, Ноябрь 2009); 3rd SeaSAR workshop “Advances in SAR Oceanography from ENVISAT, ERS and ESA third party missions” (Frascati, Italy, 25-29 January 2010); ESA Living Planet Symposium (Bergen, Norway, 28 June - 2 July 2010); 39th COSPAR Scientific Assembly (Mysore, India, 14-22 July 2012); 6th International Workshop on Science and Applications of SAR Polarimetry and Polarimetric Interferometry, POLinSAR 2013 (28 January - 1February 2013, Frascati (Rome), Italy); Asia-Pacific Conference on Synthetic Aperture Radar "Overcoming the Hardships: Responding to Disasters with SAR", (Tsukuba, Japan, 23-27 September 2013).

Результаты работы приведены в 6 статьях, опубликованных в научных журналах, входящих в перечень изданий, рекомендованных Президиумом Высшей аттестационной комиссии и в 4 патентах.

\textbf{Личный вклад автора.}

Автор работы принимал участие на всех этапах исследования от постановки задачи до анализа результатов, разрабатывал компьютерные программы, реализующие предложенные в работе методы и алгоритмы, производил обработку спутниковых данных. 

\textbf{Структура и объём диссертации.}

Диссертационная работа состоит из введения, трёх глав, заключения, библиографии, включающей 78 наименований, из них 73 на иностранных языках. Общий объём работы -- 118 машинописных страниц, включая 44 рисунка.

\textbf{Во введении} обоснована актуальность темы работы, определены цели и задачи исследования, показаны научная новизна и практическая значимость работы, сформулированы положения, выносимые на защиту.

\textbf{В первой главе} описывается метод восстановления пространственных вариаций среднеквадратичного наклона (СКН) морской поверхности по солнечному блику, регистрируемому оптическими сканерами из космоса. Разработанный метод применяется к анализу данных спутниковых оптических спектрометров MODIS и MERIS. Описываются разработанный алгоритм и программное обеспечение для восстановления СКН.

\textbf{В разделе 1.1} описаны общие представления о физике рассеяния и распространения видимого излучения в морской среде.

\textbf{В разделе 1.2} даётся обзор оптических методов исследования Океана из Космоса. В качестве примеров приводятся характеристики двух спектрометров MODIS и MERIS, данные которых использовались в работе. 

\textbf{В разделе 1.3} описывается метод восстановления пространственных вариаций среднеквадратичного наклона (СКН) морской поверхности по солнечному блику, регистрируемому оптическими сканерами из космоса.

\textbf{В разделе 1.4} рассматриваются основные технические характеристики приборов MODIS и MERIS, а также особенности формирования спутниковых оптических изображений, полученных этими сканерами.

\textbf{В разделе 1.5} описывается процедура и программа восстановления СКН по полям яркости спутниковых оптических изображений MODIS и MERIS. Приводится краткое описание языков программирования, использованных для реализации описанных алгоритмов.

\textbf{Во второй главе} метод, описанный в первой главе, применяется для исследования морской поверхности, покрытой нефтяными плёнками. Приводится совместный анализ полученных результатов с данными радиолокаторов с синтезированием апертуры (РСА), и раскрываются преимущества синергетического подхода в исследовании поверхностных сликов.

\textbf{В разделе 2.1} разработанный метод восстановления контрастов СКН применен к анализу поверхностных проявлений разливов нефти природного происхождения (так называемые «грифоны») в Мексиканском заливе.

\textbf{В разделах 2.2} и \textbf{2.3} проведён совместный анализ РЛ и оптических изображений катастрофического разлива нефти в результате взрыва на нефтяной платформе Дипвотер Хорайзон (англ. Deepwater Horizon) в Мексиканском заливе 26 Апреля 2010г.

\textbf{В третьей главе} рассматриваются примеры исследования суб- и мезомасштабной динамики Океана по оптическим и радиолокационным изображениям. При взаимодействии волн и течений СКН морской поверхности может изменяться, следовательно появляется возможность идентификации течений по изображениям солнечного блика. Используется синергетический подход для исследования поверхностных проявлений мезомасштабных течений  по оптическим (включая ИК-канал) и РСА изображениям Океана. Установлено, что аномалии «шероховатости» поверхности Океана, полученные по изображениям солнечного блика хорошо соотносятся с аномалиями на РСА изображениях. Поля аномалий «шероховатости» поверхности океана пространственно коррелируют с зонами дивергенции течений, расположенных в областях сильных градиентов температуры поверхности Океана (ТПО). Проводится анализ и интерпретация данных наблюдений на основе модельных представлений.

\textbf{В разделе 3.1} разработанный алгоритм применяется к спутниковым оптическим изображениям внутренних волн (ВВ), -- как простейшем типе течений. 

\textbf{В разделе 3.2} проводится синергетический анализ изображений MODIS и ASAR на примере района течения мыса Игольный), характеризующийся интенсивной мезомасштабной динамикой.

\textbf{В разделе 3.3} дается интерпретация данных наблюдений на основе расчета по модели формирования РЛ-изображения RIM \citep{Kudryavtsev2005,Johannessen2005}.


\clearpage