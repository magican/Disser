\chapter*{Введение}							% Заголовок
\addcontentsline{toc}{chapter}{Введение}	% Добавляем его в оглавление


Современную научно-исследовательскую деятельность в области океанологии и метеорологии, которые плохо обеспечены контактными данными, уже невозможно представить без использования данных спутниковых наблюдений. Существующие методы обработки и анализа спутниковых измерений обеспечивают получение данных об «огромном» наборе параметров, характеризующих систему «атмосфера -- земля -- океан», которые в настоящее время широко используются в задачах мониторинга и прогноза окружающей среды. В настоящее время ряд спутниковых геофизических продуктов находится в открытом доступе. В приложении к океанографическим исследованиям, спутниковые данные могут быть получены, например, из центра данных Physical Oceanography Distributed Active Archive Center (PO.DAAC) -- американский центр НАСА (http://podaac.jpl.nasa.gov/), Centre ERS d'Archivage et de Traitement -- (CERSAT) -- французский центр данных института IFREMER (http://cersat.ifremer.fr/); информационный портал спутниковых данных РГГМУ (SATIN - http://satin.rshu.ru/).

В то же время, перспективы развития исследований Земли из Космоса неизбежно требуют создания новых подходов и методов обработки, анализа и использования спутниковой информации. Оптические методы исследования Земли являются наиболее развитыми и широко используемыми в оперативной практике. В настоящее время на орбите Земли находится большое количество сканеров, работающих в оптическом диапазоне (например, сканеры MODIS на спутниках Terra и Aqua, радиометры AVHRR на серии спутников NOAA). Одно из основных применений данных оптических сканеров, - изучение «цвета» Океана (содержание фитопланктона и минеральной взвеси, биогеохимические характеристики), а также температуры его поверхности.

При изучение оптических характеристик Океана, солнечная радиация, отраженная от морской поверхности, является шумом по отношению к радиации рассеянной в верхнем слое Океана. В областях солнечного блика отражённая радиация составляет значительную часть регистрируемого излучения, что исключает возможность применения алгоритмов восстановления «цвета» Океана. Существование солнечного блика приводит к тому, что огромная часть спутниковых сканерных данных (до 30\%) не может быть использована в классических океанографических приложениях. Области, где восстановление параметров цвета Океана по спутниковым данным невозможно, маскируется для конечного пользователя и, таким образом, «выбрасывается в мусорный ящик». Пример маскировки областей поверхности океана, «засвеченных» солнечным бликом в данных сканера MODIS приведен на Рис. 1.



Рис. 1. «Глобальная» маска блика от 21 Марта 2004 года для MODIS/Aqua.



Повышение информативности данных измерений спутниковыми сканерами определяет \textbf{актуальность} данного исследования. Основная идея состоит в том, что отраженная солнечная радиация несёт информацию о характеристиках «шероховатости» поверхности Океана. В этом случае данные оптических сканеров могут быть использованы для исследования статистических характеристик ветрового волнения и их вариаций, вызванных различными океаническими процессами. Актуальность данного исследования определяется так же необходимостью разработки нового метода, позволяющего использовать отбрасываемые ранее данные оптических сканеров для исследования проявления различных динамических процессов на поверхности Океана. Предлагаемый подход, совместно с существующими радиолокационными (РЛ) методами наблюдения поверхности Океана, открывает новые возможности для мониторинга океанических явлений из Космоса по их поверхностным проявлениям.

\textbf{Основной целью работы} является разработка метода исследования поверхности Океана по спутниковым изображениям солнечного блика, и применение этого метода для исследования нефтяных загрязнений и поверхностных проявлений динамических процессов в Океане.

Для достижения поставленной цели в работе решаются следующие \textbf{задачи}:

\begin{enumerate}
\item  разработать метод восстановления пространственных вариаций среднеквадратичного наклона (СКН) морской поверхности по полю яркости солнечного блика;

\item  применить разработанный метод для анализа данных спутниковых оптических сканеров MODIS и MERIS;

\item  исследовать поверхностные проявления биологических и нефтяных сликов в солнечном блике и в поле СКН морской поверхности, а также исследовать подобие и отличия аномалий «шероховатости» морской поверхности в сликах, измеряемых оптическими и радиолокационными методами;

\item  исследовать особенности проявления внутренних волн и мезомасштабных течений на морской поверхности по изображениям солнечного блика;

\item  исследовать связь аномалий характеристик «шероховатости» морской поверхности с параметрами мезомасштабных течений на основе синергетического анализа оптических и радиолокационных изображений;

\item  разработать специализированное программно-математическое обеспечение, сопровождающее разработанные методы.
\end{enumerate}

 Научная новизна

Научная новизна работы состоит в следующем:

\begin{enumerate}
\item  разработан новый метод восстановления пространственных вариаций СКН морской поверхности по изображениям солнечного блика, регистрируемого спутниковыми оптическими сканерами;

\item  исследованы проявления нефтяных сликов в солнечном блике и в поле СКН морской поверхности. Показано что контрасты СКН в нефтяных сликах систематически ниже контрастов СКН в сликах биологического происхождения, эффективный коэффициент упругости тонкой нефтяной плёнки может быть задан как E=15мН/м;

\item  продемонстрировано, что наблюдения Океана в солнечном блике могут являться эффективным инструментом исследования ВВ. Поверхностные проявления ВВ видны через модуляцию среднеквадратичного наклона морской поверхности. Усиление среднеквадратичного наклона (СКН) происходит в зонах конвергенции течения ВВ, а его подавление - в зонах дивергенции;

\item  на основе совместного анализа оптических и радиолокационных изображений установлено, что мезомасштабные течения проявляются на морской поверхности в виде пространственных вариаций СКН и обрушений волн;

\item  проявления мезомасштабных течений обусловлено в основном влиянием дивергенции течений на ветровые волны. Соответственно, аномалии состояния поверхности в виде увеличения/уменьшения ее СКН и интенсивности обрушений волн привязаны к зонам конвергенции/дивергенции течений, которые в свою очередь связаны с градиентами поля завихренности квази-геострофического течения.
\end{enumerate}

 Практическая и научная значимость

Полученные научные результаты реализованы в виде алгоритмов и элементов программного обеспечения для обработки РСА и оптических изображений, а также восстановления статистических параметров поверхности Океана.

Предложенные алгоритмы и методики были апробированы и внедрены в Международном центре по окружающей среде и дистанционному зондированию им. Нансена (NIERSC), а также в Лаборатории Спутниковой Океанографии (ЛСО, на англ. SOLab) РГГМУ, в виде элементов спутникового информационного портала SATIN (от англ. SATellite Data Search and Manage INformation Portal), для поиска, получения, отображения, распространения и хранения данных дистанционного зондирования (http://satin.rshu.ru/), а также как элемент разрабатываемой синергетической платформы SYNTool (http://syntool.solab.rshu.ru/) ЛСО РГГМУ.

В результате применения разработанных методов и алгоритмов, получена возможность использовать данные о яркости поверхности Океана внутри солнечного блика, для исследования океанографических явлений по их поверхностным проявлениям, что позволило значительно расширить область применимости оптических сканеров. Показано что, применение синергетического подхода, основанного на совместном использовании РСА и оптических данных, позволяет лучше понять механизмы проявления океанических явлений на поверхности и выработать предложения по комбинации датчиков и спектральных каналов для повышения эффективности спутникового мониторинга морской среды 

 Положения, выносимые на защиту

\begin{enumerate}
\item  метод диагностики пространственных аномалий «шероховатости» поверхности Океана по изображениям солнечного блика и его применение к данным измерений спутниковыми сканерами;

\item  количественные характеристики контрастов СКН в нефтяных сликах и их отличие от сликов биологического происхождения, что с привлечением РСА данных открывает возможность создания метода идентификации антропогенных загрязнений Океана и определения их параметров (площадь, объем вещества);

\item  особенности проявления внутренних волн и мезомасштабные течений на морской поверхности в виде пространственных аномалий среднеквадратичного наклона поверхности, спектра коротких ветровых волн и обрушений ветровых волн, идентифицируемых спутниковыми оптическими сканерами и РСА; 

\item  связь аномалий характеристик ветрового волнения (СКН, обрушения) с зонами дивергенции течений, и их пространственная привязка к областям сильных градиентов завихренности полей квази-геострофических течений.
\end{enumerate}

 Апробация работы и публикации



Результаты работы докладывались на различных международных конференциях и семинарах, в частности на7th All russian Opened Conference ``Modern Problems of Earth Remote Sensing from Space'' (Moscow, Russia, November 2009); 3rd SeaSAR workshop ``Advances in SAR Oceanography from ENVISAT, ERS and ESA third party missions'' (Frascati, Italy, 25-29 January 2010); ESA Living Planet Symposium (Bergen, Norway, 28 June - 2 July 2010); 39th COSPAR Scientific Assembly (Mysore, India, 14-22 July 2012); 6th International Workshop on Science and Applications of SAR Polarimetry and Polarimetric Interferometry, POLinSAR 2013 (28 January - 1February 2013, Frascati (Rome), Italy); Asia-Pacific Conference on Synthetic Aperture Radar "Overcoming the Hardships: Responding to Disasters with SAR", (Tsukuba, Japan, 23-27 September 2013).

Результаты работы приведены в 6 статьях, опубликованных в научных журналах, входящих в перечень изданий, рекомендованных Президиумом Высшей аттестационной комиссии и в 4 патентах.

 Личный вклад автора

Автор работы принимал участие на всех этапах исследования от постановки задачи до анализа результатов, обеспечивал разработку компьютерных программ, реализующих предложенные в работе методы и алгоритмы, производил обработку спутниковых данных. 

 Структура и объём диссертации

Диссертационная работа состоит из введения, трёх глав, заключения, библиографии, включающей 85 наименований, из них 79 на иностранных языках. Общий объём работы -- 118 машинописных страниц, включая 44 рисунка.



 Содержание Работы

Во введении обоснована актуальность темы работы, определены цели и задачи исследования, показаны научная новизна и практическая значимость работы, сформулированы положения, выносимые на защиту.

В первой главе описывается метод восстановления пространственных вариаций среднеквадратичного наклона (СКН) морской поверхности по солнечному блику, регистрируемому оптическими сканерами из космоса. Разработанный метод применяется к анализу данных спутниковых оптических спектрометров MODIS и MERIS. Описываются разработанный алгоритм и программное обеспечение для восстановления СКН.

В разделе 1.1 описаны общие представления о физике рассеяния и распространения видимого излучения в морской среде.

В разделе 1.2 даётся обзор оптических методов исследования Океана из Космоса. В качестве примеров приводятся характеристики двух спектрометров MODIS и MERIS, данные которых использовались в работе. Известно, что отражённый от морской поверхности солнечный свет составляет основной вклад восходящей радиации и создаёт значительные трудности для разработчиков алгоритмов восстановления цвета Океана. Однако, в солнечном блике содержится ценная информация о статистических характеристиках шероховатости морской поверхности, её среднеквадратичном наклоне (СКН), асимметрии и эксцессе, как было впервые показано в работе Кокса и Манка [Cox, Munk, 1954], а также применительно к спутниковым данным - Бреоном и Хенриотом [Brйon, Henriot, 2006]. 

В разделе 1.3 описывается метод восстановления пространственных вариаций среднеквадратичного наклона (СКН) морской поверхности по солнечному блику, регистрируемому оптическими сканерами из космоса. Вариации СКН связаны с поверхностными проявлениями различных процессов, происходящих в верхнем слое Океана, например, искусственными и биологическими сликами, внутренними волнами, границами мезо-масштабных течений и фронтальными разделами. С этой точки зрения, предложенный метод может рассматриваться как определенный шаг в направлении развития методов диагностики Океана из Космоса.

Приводятся основные соотношения модели изображения морской поверхности в области солнечного блика и связи вариаций среднеквадратичного наклона (СКН) $\widetilde{s^{2} }$ с вариациями яркости морской поверхности $\widetilde{B}$:

\begin{equation} \label{ZEqnNum404712} \frac{\tilde{B}}{\overline{B}} =-T\frac{\widetilde{s^{2} }}{s^{2} } ,  \end{equation} 

где $T$ -- это передаточная функция:

\begin{equation} \label{ZEqnNum527172} T=1+\frac{1}{2} \left(\frac{\xi }{p} \frac{\partial p}{\partial \xi } +\frac{\eta }{p} \frac{\partial p}{\partial \eta } \right) \end{equation} 

$\xi =Z_{x} /s$ и $\eta =Z_{y} /s$, $p$ - -- ``безразмерная'' плотность вероятностей наклонов морской поверхности, нормированная на среднеквадратичный наклон морской поверхности $s^{2} $, $Z_{x} $ и $Z_{y} $ -- наклоны морской поверхности, удовлетворяющие условиям зеркального отражения солнечного излучения в приемную апертуру прибора (см. соотношения (1-4) в работе [Cox, Munk, 1954]).

При наличии 2D поля яркости солнечного блика (как, например, в данных сканера MODIS) передаточная функция $T$ может быть определена по осредненным градиентам яркости, которые берутся непосредственно из изображения солнечного блика. Градиенты безразмерной плотности вероятности наклонов p находятся из градиентов крупномасштабной яркости солнечного блика следующим образом:

\begin{equation} \label{3)} \begin{array}{l} {\frac{\xi }{p} \frac{\partial p}{\partial \xi } =\frac{\xi }{B} \frac{\partial B}{\partial \xi } -\frac{4Z_{x}^{2} }{1+Z_{x}^{2} +Z_{y}^{2} } } \\ {\frac{\eta }{p} \frac{\partial p}{\partial \eta } =\frac{\eta }{B} \frac{\partial B}{\partial \eta } -\frac{4Z_{y}^{2} }{1+Z_{x}^{2} +Z_{y}^{2} } } \end{array} \end{equation} 

Градиент яркости в пространстве ($\xi ,\eta $) можно выразить через градиенты поля в двух ортогональных направлениях ($\nabla _{l} B$ и $\nabla _{n} B$) как

\begin{equation} \label{4)} \begin{array}{l} {\frac{\xi }{B} \frac{\partial B}{\partial \xi } =Z_{x} \frac{\nabla _{l} \ln (B\cos \theta _{v} )\cdot \nabla _{n} Z_{y}^{} -\nabla _{n} \ln B\cdot \nabla _{l} Z_{y}^{} }{\Delta } } \\ {\frac{\eta }{B} \frac{\partial B}{\partial \eta } =Z_{y} \frac{\nabla _{n} \ln (B\cos \theta _{v} )\cdot \nabla _{l} Z_{x}^{} -\nabla _{l} \ln B\cdot \nabla _{n} Z_{x}^{} }{\Delta } } \end{array},  \end{equation} 

где лапласиан $\Delta =\nabla _{l} Z_{x} \cdot \nabla _{n} Z_{y}^{} -\nabla _{n} Z_{x} \cdot \nabla _{l} Z_{y}^{} $. В применении к данным сканера MODIS $\nabla _{l} B$ соответствовало градиенту в направлении сканирования, а $\nabla _{n} B$ - в перпендикулярном направлении (вдоль полета спутника). 

В ряде случаев поля яркости морской поверхности имеют характер 1D поля (как например в данных сканера MERIS) и для определение $T$ необходимо задание модели плотности распределения вероятности морских уклонов. Для Гауссова распределения вероятности наклонов передаточная функция принимает следующий вид:

\begin{equation} \label{5)} T=1-\alpha _{2} \xi ^{2} +\alpha _{12} \xi \eta -\alpha _{1} \eta ^{2} ,  \end{equation} 

где коэффициенты $a_{1} $, $a_{2} $ и $a_{12} $ зависят от направления ветра и анизотропии наклонов поверхности; для изотропного распределения наклонов: $a_{1} =a_{2} =1$ и $a_{12} =0$. 

В разделе 1.4 рассматриваются основные технические характеристики приборов MODIS и MERIS, а также особенности формирования спутниковых оптических изображений, полученных этими сканерами. Показаны их отличия и сходства в области солнечного блика. Обосновано, почему в работе используются данные уровня 1B в красном канале.

В разделе 1.5 описывается процедура и программа восстановления СКН по полям яркости спутниковых оптических изображений MODIS и MERIS. Приводится краткое описание языков программирования, использованных для реализации описанных алгоритмов. Описывается пошаговая процедура применения разработанного метода, включая этапы чтения, обработки и восстановления СКН на примере спутниковых снимков MODIS и MERIS района течения Гольфстрим у берегов Флориды. Дается информация об использовании вспомогательных данных для обработки изображений, включая файлы с геолокацией и геометрией съёмки, а также вспомогательные данные о приводном ветре и маске Земли. 

Во второй главе метод, описанный в первой главе, применяется для исследования морской поверхности, покрытой нефтяными плёнками. Приводится совместный анализ полученных результатов с данными радиолокаторов с синтезированием апертуры (РСА), и раскрываются преимущества синергетического подхода в исследовании поверхностных сликов.

В разделе 2.1 разработанный метод восстановления контрастов СКН применен к анализу поверхностных проявлений разливов нефти природного происхождения (так называемые «грифоны») в Мексиканском заливе. Показано, что вариации яркости блика, связанные с присутствием нефтяных сликов, бывают как положительными, так и отрицательными. Дается описание явления инверсии контрастов в солнечном блике, и приводятся причины его возникновения. Представлена зависимость контрастов СКН от скорости ветра по серии выбранных нефтяных сликов. Делается вывод, что контрасты СКН в нефтяных сликах систематически ниже контрастов СКН в биологических сликах. 

В разделе 2.2 проанализирован катастрофический разлив нефти в результате взрыва на нефтяной платформе Дипвотер Хорайзон (англ. Deepwater Horizon) в Мексиканском заливе. Предложенный подход применён к изображениям солнечного блика, полученным приборами MODIS (MODIS/Terra, Май, 24, 2010, 16:45 GMT) и MERIS (MERIS/Envisat, Май 24, 2010, 16:17 GMT). На Рис.~2 приведены контрасты СКН $\widetilde{s}^{2} /s_{0}^{2} $, восстановленные по изображениям яркости в солнечном блике.

Рис.~2. Аномалии СКН $\widetilde{s^{2} }/s^{2} $, восстановленные по данным прибора MERIS (вверху) и MODIS (внизу). Красными стрелками обозначены области инверсии контрастов, где восстановление контрастов СКН невозможно. Яркие участки контрастов СКН, обведённые жёлтым контуром, не относятся к особенностям шероховатости морской поверхности, а, скорее всего, отражают  оптические свойства (цвет) толстой нефтяной плёнки, толщина которой значительно большей длины волны красного света.



В результате анализа показано, что разработанные алгоритмы дают возможность оценить пространственное распределение нефтяных загрязнений по полям контрастов СКН. Обнаружено, что в некоторых областях нефтяных загрязнений толщина нефтяной плёнки значительно превышает длину волны красного света. В этих областях, восстановленные значения аномалий СКН противоречат ожидаемому эффекту подавления коротковолнового волнения в сликах, а контрасты яркости поверхности определяются оптическими свойства нефти (её цветом).

В разделе 2.3 рассмотрен случай синхронной съёмки нефтяного разлива 26 Апреля 2010г. приборами MERIS (15:58 GMT) и ENVISAT ASAR (15:56 GMT), представленный на Рис.~3

Синхронные ASAR РСА и оптические изображения дают возможность проанализировать подобие и различия между проявлениями одних и тех же нефтяных сликов. Показано, что оптические и РЛ-контрасты одного и того же слика, сформированного тонкой нефтяной плёнкой, хорошо коррелируют. При этом радиолокационные контрасты примерно в 1.6 раза сильнее контрастов СКН. Для интерпретации наблюдаемых контрастов УЭПР и СКН в сликах использовалась модель спектров коротких волн, предложенная в работах [Kudryavtsev и др., 2005; Yurovskaya и др., 2013]. Получено, что модельные контрасты согласуются с наблюдениями в том случае, если эффективный коэффициент упругости равен E=15мН/м.

Показано, что при совместном анализе контрастов яркости морской поверхности в солнечном блике и РСА контрастов в сликах, можно сделать качественное заключение о толщине пленки. Желтым контуром на Рис.~3очерчена область слика, в которой толщина нефтяной пленки существенно превышает длину волны красного цвета. В этом случае, контрасты УЭПР отрицательны (в силу подавления коротких волн), а восстановленные значения контрастов СКН -- имеют «нереалистические» положительные значения, обусловленные цветом нефти.

Рис.~3. Увеличенные фрагменты изображений ASAR (слева) и MERIS (справа). Изображения содержат нефтяной разлив и представлены в терминах контрастов УЭПР (линейные единицы) и контрастов СКН. Толщина нефтяной плёнки в области внутри жёлтого контура значительно больше длины волны красного света.



В третьей главе рассматриваются примеры исследования суб- и мезомасштабной динамики Океана по оптическим и радиолокационным изображениям. При взаимодействии волн и течений СКН морской поверхности может изменяться, следовательно появляется возможность идентификации течений по изображениям солнечного блика. Используется синергетический подход для исследования поверхностных проявлений мезомасштабных течений  по оптическим (включая ИК-канал) и РСА изображениям Океана. Установлено, что аномалии «шероховатости» поверхности Океана, полученные по изображениям солнечного блика хорошо соотносятся с аномалиями на РСА изображениях. Поля аномалий «шероховатости» поверхности океана пространственно коррелируют с зонами дивергенции течений, расположенных в областях сильных градиентов температуры поверхности Океана (ТПО). Проводится анализ и интерпретация данных наблюдений на основе модельных представлений.

В разделе 3.1 разработанный алгоритм применяется к спутниковым оптическим изображениям внутренних волн (ВВ), -- как простейшем типе течений. В качестве примера выбран район западно-экваториальной Атлантики, напротив устья реки Амазонки, - область регулярного возникновения интенсивных внутренних волн (ВВ), формируемых полусуточными приливными волнами (см. например, [Ivanov, Pelinovsky, Talipova, 1993]). Изображение MODIS/Aqua этого района, полученное 26 апреля 2009, 16:20, показано на Рис. 4~а. Для иллюстрации работы метода, на этом рисунке приведены: (а) -- исходное изображение, (б) -- вариации яркости блика, (в) -- передаточная функция (см. ур-ние \eqref{ZEqnNum527172}) и (г) восстановленное поле контрастов СКН по уравнению \eqref{ZEqnNum404712}. В исходном изображении и в поле СКН отчетливо наблюдаются следующие друг за другом уединенные ВВ.

Профиль вариаций СКН вдоль сечения A-B (отмеченного на Рис. 4~в) представлен на Рис 5~вверху. Проявление ВВ в вариациях СКН имеет «биполярную» форму, с положительными и отрицательными аномалиями СКН. Для интерпретации наблюдений, предположим, что увеличение/уменьшение СКН морской поверхности связано с конвергенцией/дивергенцией поверхностных течений, вызванных ВВ, т.е. $K_{s} \equiv {\widetilde{s}^{2} \mathord{\left/ {\vphantom {\widetilde{s}^{2}  s_{0}^{2} }} \right. \kern-\nulldelimiterspace} s_{0}^{2} } \propto {\partial u\mathord{\left/ {\vphantom {\partial u \partial x}} \right. \kern-\nulldelimiterspace} \partial x} $. В этом выражении коэффициент пропорциональности является «подгоночной» константой, которая определяется сравнением наблюдаемым и модельным СКН. Тогда, поверхностная скорость определяется как $u(x)=c_{u} \int _{0}^{x}\left(K_{s} -\left\langle K_{s} \right\rangle \right)dx $ Определенная таким образом поверхностная скорость течений, задавалась в качестве входного параметра для модельных расчетов поверхностных проявлений ВВ по модели формирования РЛ-изображения RIM (от англ. Radar Imaging Model) [Kudryavtsev и др., 2005]. Далее, константа $c_{u} $ определялась так, чтоб отклонение модельных расчетов от наблюдений было минимальным. Результаты расчетов представлены на Рис 5. Так как «профиль» модельных контрастов согласуется с измеренными вариациями СКН, можно заключить, что наблюдаемые модуляции СКН, в действительности, определяются конвергенцией и дивергенцией течений, индуцируемой ВВ на поверхности. На Рис 5 так же представлены смещения термоклина, соответствующие «восстановленному» полю скорости, которые согласуются с измерениями ВВ в этом районе [Кудрявцев и др., 2008].

В разделе 3.2 проводится синергетический анализ изображений MODIS и ASAR на примере района течения мыса Игольный), характеризующийся интенсивной мезомасштабной динамикой, отчетливо видимой в изменчивости поля ТПО на . Исследования проводятся на фоне сильно меняющегося поле ветра, восстановленного по РСА, . 

Рис. 4. (а) Фрагмент исходного изображения MODIS/Aqua (26 апреля 2009, 16:20) в красном канале района устья реки Амазонки с проявлениями ВВ. (б) Контрасты яркости $\widetilde{B}/\overline{B}$. (в) Передаточная функция. (г) Контрасты СКН ${\widetilde{s^{2} }\mathord{\left/ {\vphantom {\widetilde{s^{2} } s^{2} }} \right. \kern-\nulldelimiterspace} s^{2} } $. Белые области на изображениях -- маска облаков. Линия A-B обозначает положение сечения, показанного на Рис 5.



Рис 5. Профиль контрастов СКН и RIM-моделирование сечения А-В, проходящего через ВВ на Рис. 4~в. (вверху) Профиль контрастов СКН (сплошная линия) вдоль сечения A-B. Пунктирная линия показывает результаты модельных расчетов. (внизу) Восстановленные смещения термоклина во внутренних волнах (жирная линия) и скорость течения на поверхности, вызванное ВВ (тонкая линия).





Рис. 6. Поле ветра и поле поверхностной температуры Океана. 

(а) Поле ветра, полученное по изображению ASAR WS, с использованием алгоритма CMOD4. Направление ветра взято из модели NCEP. (б) поле ТПО, полученное по данным MODIS. Белые области - маска облаков. Юг Африканского континента выделен коричневым цветом.



На Рис.~7 представлено восстановленное поле контрастов СКН, и контрасты УЭПР РСА. На обоих полях видны проявления фронтального раздела, меандров и вихрей. Типичные значения амплитуды контрастов СКН, наблюдаемые на изображении, имеют величины около 20-30\%  с масштабами проявления  от 1 до 10км. Сопоставляя поля контрастов СКН и УЭПР РСА, можно видеть пространственную «корреляцию» проявления одних и тех же мезомасштабных явлений, что является «неожиданным», поскольку физика формирования оптических и радиолокационных изображений весьма различна.

Рис.~7. (а) Поле контрастов СКН, полученное из поля вариаций яркости, с применением разработанного алгоритма и поле контрастов УЭПР РСА (б)



Для выяснения закономерности формирования поверхностных проявлений, необходимо иметь информацию о мезомасштабных особенностях полей течений. Для восстановление мезомасштабной изменчивости поля скорости геострофического течения (ГТ) используется модель, предложенная [Isern-Fontanet и др., 2008]. В рамках этой модели функция тока ГТ $\widehat{\psi }({\it k},z)$ и поле ТПО $\widehat{T}_{s} ({\it k})$ в пространстве Фурье связаны следующим соотношением:

\begin{equation} \label{6)} \widehat{\psi }({\it k},z)=\frac{g\alpha \widehat{T}_{s} ({\it k})}{fn_{b} k} \exp (n_{0} kz),  \end{equation} 

где $n=N/f$ отношение Прандтля для частот Брента-Вяйсяля $N_{0} $ и $N_{b} $, определяющих, соответственно, мезо- и субмезомасштабные свойства потока. Поле скорости ГТ определяется через функцию тока $\widehat{\psi }$, как $\widehat{{\it U}}=(-ik_{y} \widehat{\psi },ik_{x} \widehat{\psi })$, или в физическом пространстве как ${\it U}=(-\partial \psi /\partial x_{2} ,\partial \psi /\partial x_{1} )$. На Рис.~8 приведено поле завихренности ГТ, восстановленное по полю ТПО, показанному на .

Однако, поле ГТ является бездивергентным. Как показано в [Kudryavtsev и др., 2005], влияние такого типа течений на интегральные параметры спектра ветровых волн (СКН, обрушения ветровых волн) является слабым, т.е. проявления ГТ на поверхности океана должны быть «невидимы». Поэтому, наблюдаемые на Рис.~7 поверхностные проявления, скорее всего, связаны с другими факторами. 

Взаимодействие ветрового дрейфа с полем скорости ГТ может приводит к генерации достаточно сильной вторичной агеострофической циркуляции [Garrett, Loder, 1981]. Следуя модели [Klein, Hua, 1990] дивергенция агеострофической компоненты течения, генерируемого при взаимодействии Экмановского течения с ГТ, имеет вид:

\begin{equation} \label{7)} \nabla \cdot {\it u}=-f^{-1} u_{\beta }^{e} \frac{\partial }{\partial x_{\beta } } \Omega ,  \end{equation} 

где $\Omega _{z} =\partial U_{2} /\partial x_{1} -\partial U_{1} /\partial x_{2} \equiv \Delta \psi $ - завихренность КГТ, $f$ - параметр Кориолиса, $u_{\beta }^{e} $ - компоненты Экмановской скорости течения.





Рис.~8. Завихренность поля КГТ и дивергенция поверхностного течения, восстановленные по наблюдаемому полю ТПО. (а) Завихренность поверхностного квазигеострофического течения (КГТ), полученное по полю ТПО, показанному на ~б. (б) Поле поверхностной дивергенции, $\nabla \cdot {\it u}$, связанное с вторичной агеострофической циркуляцией, генерируемой  в результате взаимодействия Экмановского течения  с КГТ. Поле дивергенции инвертировано (показано поле $-\nabla \cdot {\it u}$), поэтому яркие области на изображении соответствуют зонам конвергенции, а тёмные -- зонам дивергенции.



Дивергенция поверхностных течений, рассчитанная по полю завихренности (Рис.~8) и Экмановскому дрейфу, соответствующему полю ветра на 6~а приведено на Рис.~8~б. Видно, что зоны дивергенции/конвергенции привязаны к зонам градиентов завихренности КГТ, и соответственно -- к зонам резких перепадов ТПО.

Сопоставляя поля контрастов СКН и УЭПР на Рис.~~а с полями дивергенции поверхностных течений на Рис.~8~б (см. также Рис.~9), можно прийти к выводу, что усиление/ослабление поверхностной шероховатости происходит в зонах конвергенции/дивергенции поверхностных течений. Фрагмент поля дивергенции и контрастов РСА, представленный на Рис.~9, наиболее отчетливо демонстрирует этот факт. Заметим, что основным механизмом формирования поверхностных проявлений течений на РСА изображениях является модуляция обрушений ветровых волн. Поэтому, контрасты РСА изображения на рис 8 отражают связь пространственных вариаций обрушений волн с полем дивергенции мезомасштабных течений. 

Рис.~9. (a) Фрагмент изображения контрастов УЭПР РСА, полученное по изображению ASAR WS (18го Ноября 2007, 7:24 GMT), и (b) соответствующий фрагмент поля дивергенции поверхностного течения по данным MODIS (18го Ноября, 2007, 12:05 GMT). Яркие области на рисунке (b) соответствуют зонам конвергенции, а тёмные -- дивергенции течения.



Хорошее соответствие между аномалиями шероховатости поверхности Океана и градиентами ТПО может быть рассмотрено в качестве «экспериментального подтверждения» того факта, что влияние дивергенции поверхностного течения на ветровые волны является основным механизмом формирования поверхностных проявлений мезомасштабных течений. 



В разделе 3.3 дается интерпретация данных наблюдений на основе расчета по модели RIM [Johannessen и др., 2005; Kudryavtsev и др., 2005]. Результаты моделирование проявления поля поверхностного течения в полях СКН, обрушения ветровых волн и УЭПР приводятся на Рис.~10. Пространственное распределение контрастов параметров поверхности подобны полю конвергенции поверхностного течения. Модельные расчеты подтверждают обнаруженный экспериментально факт, что доминирующим параметром поверхностных течений, определяющим их появление, является его дивергенция вектора скорости. 



Рис.~10 -- Фрагменты восстановленного по ТПО поля дивергенции поверхностного течения и смоделированных характеристик: контрастов СКН, обрушений волн, и контрастов УЭПР. (а) Дивергенция поля поверхностного течения, полученное по полю ТПО (~б). Яркие области на рисунке (а) соответствуют конвергенции течения, а тёмные -- дивергенции. На другие рисунках представлены модельные расчеты проявления вихревого образования в полях контрастов СКН (б), обрушенияй ветровых волн (в), и контрастов УЭПР (г).


\clearpage