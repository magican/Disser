\chapter*{Заключение}						% Заголовок
\addcontentsline{toc}{chapter}{Заключение}	% Добавляем его в оглавление

В данной работе предложен метод....

Обычно солнечный боик выбрасывается....

Конкретные результаты работы заключаются в следующем:

\begin{itemize}

\item Предложен метод количественной интерпретации пространственных вариаций СКН морской поверхности по солнечному блику, регистрируемому оптическими сканерами из космоса. Помимо выявления зависимости от скорости ветра, разработанный метод открывает дополнительные возможности исследовать поверхностные явления на море (как например слики или особенности течений), приводящие к вариациям СКН, а как следствие, к изменениям яркости на внутренних масштабах солнечного блика, т.е. на масштабах много меньших «ширины» самого блика.

\item Разработан алгоритм восстановления СКН по спутниковым оптическим изображениям солнечного блика. Алгоритм восстановления СКН использует передаточную функцию, связывающую контрасты яркости в солнечном блике с контрастами СКН. Определение передаточной функции напрямую основано на наблюдаемых градиентах яркости солнечного блика, без априорного здания плотности распределения уклонов, что позволяет работать с различными оптическими спектрометрами. На основе предложенного метода разработаны два алгоритма восстановления СКН: для 2D и 1D полей яркости.

\item Написано соответствующее программное обеспечение для восстановления СКН по полям яркости. Разработаны процедуры загрузки, чтения и обработки спутниковых снимков приборов MODIS и MERIS, а также вспомогательной информации. Программы реализованы на языках программирования Matlab и Python.

\item Представлен метод восстановления поля скорости течения на масштабах от 30 до 300 километров по изображениям ТПО. На мезомасштабах (10-500км) и субмезомасштабах (1-10км), динамика океана такова, что, зачастую, в равновесно стратифицированном, быстро вращающемся потоке горизонтальные скорости, в среднем, значительно больше вертикальных. Таким образом, такое движение можно рассматривать как квази-двумерное, и его изучение проводить в рамках некоторых приближений. Разработан практический подход для восстановления поля скорости течения на масштабах от 30 до 300 километров по изображениям ТПО.

\item Реализован алгоритм восстановления квазигеострофической и агеострофической циркуляции по температуре поверхности океана. В работе предполагается, что полное поле океанического течения можно представить в виде суммы квазигеострофического течения (КГТ) , ветрового течения  (которое также может включать инерционное течение), и вторичной агеострофической циркуляции (ВАЦ) , в результате взаимодействия Экмановского течения с КГТ и диабатического перемешивания в слое Экмана. Основываясь на поверхностной квазигеострофической (ПКГ, от англ. SQG) динамике, выведено уравнение, напрямую связывающее дивергенцию поверхностного течения с полем ТПО.

\item Для выполнения поставленных задач было создано соответствующее программное обеспечение для восстановления поля дивергенции и завихренности поля поверхностного течения по спутниковым данным температуры поверхности океана.

\end{itemize}

Какая-нибудь заключающая фраза \ldots

\clearpage