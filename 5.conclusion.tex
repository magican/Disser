\chapter*{Заключение}						% Заголовок
\addcontentsline{toc}{chapter}{Заключение}	% Добавляем его в оглавление


При изучение оптических характеристик Океана солнечная радиация, отраженная от морской поверхности, является шумом по отношению к радиации рассеянной в верхнем слое Океана. В областях солнечного блика отражённая радиация составляет значительную часть регистрируемого излучения, что исключает возможность применения алгоритмов восстановления «цвета» Океана. Отраженная солнечнечная радиация несёт информацию о характеристиках «шероховатости» поверхности океана. В этом случае данные оптических сканеров могут быть использованы для исследования статистических характеристик ветрового волнения и их вариаций, вызванных различными океаническими процессами. 

В данной работе предложен метод восстановления пространственных вариаций среднеквадратичного наклона (СКН) морской поверхности по солнечному блику, регистрируемому оптическими сканерами из космоса.

Разработанный метод применяется к анализу данных спутниковых оптических спектрометров MODIS и MERIS, совместно с существующими РЛ методами наблюдения поверхности Океана, для исследования нефтяных загрязнений и поверхностных проявлений динамических процессов в Океане.

Конкретные результаты работы заключаются в следующем:

\begin{itemize}

\item Разработан алгоритм восстановления СКН по спутниковым оптическим изображениям солнечного блика. Алгоритм восстановления СКН использует передаточную функцию, связывающую контрасты яркости в солнечном блике с контрастами СКН. Определение передаточной функции напрямую основано на наблюдаемых градиентах яркости солнечного блика, без априорного здания плотности распределения уклонов, что позволяет работать с различными оптическими спектрометрами. На основе предложенного метода разработаны два алгоритма восстановления СКН: для 2D и 1D полей яркости.

\item Предложен метод количественной интерпретации пространственных вариаций СКН морской поверхности по солнечному блику, регистрируемому оптическими сканерами из космоса. Помимо выявления зависимости от скорости ветра, разработанный метод открывает дополнительные возможности исследовать поверхностные явления на море (как например слики или особенности течений), приводящие к вариациям СКН, а как следствие, к изменениям яркости на внутренних масштабах солнечного блика, т.е. на масштабах много меньших «ширины» самого блика.

\item Написано соответствующее программное обеспечение для восстановления СКН по полям яркости. Разработаны процедуры загрузки, чтения и обработки спутниковых снимков приборов MODIS и MERIS, а также вспомогательной информации. Программы реализованы на языках программирования Matlab и Python.

\item Метод восстановления контрастов среднеквадратичного наклона (СКН) применён к анализу проявления нефтяных сликов естественного и техногенного происхождения по изображениям солнечного блика.

\item Установлено, что контрасты СКН в нефтяных сликах систематически ниже контрастов СКН в сликах биологического происхождения. Этот результат объясняется различием упругостей нефтяных плёнок и плёнок биологического происхождения. Показано, что эффективный коэффициент упругости для тонкой нефтяной плёнки может быть задан как как E=15мН/м.

\item Показано, что оптические и РЛ-контрасты одного и того же слика, сформированного тонкой нефтяной плёнкой, хорошо коррелируют. При этом контрасты УЭПР примерно в 1.6 раза сильнее контрастов СКН.

\item Продемонстрировано, что поверхностные проявления ВВ хорошо видны в модуляциях уклонов морской поверхности. Это связано с усилением среднеквадратичного наклона (СКН) в зонах конвергенции течения ВВ, в то время как подавление наблюдается в зонах дивергенции.

\item Предложен синергетический подход для идентификации, восстановления и анализа параметров поверхностных проявлений мезо-масштабных океанических течений по оптическим и радиолокационным изображениям, получаемым из космоса.

\item В рамках предложенного подхода, поля геострофических течений (ГТ) и вторичных агеострофических течений, с которыми связаны зоны конвергенции и дивергенции могут быть восстановлены по спутниковым полям ТПО полям РСА-ветра. 

\item Обнаружено, что поверхностные проявления мезомасштабных течений в виде аномалий СКН и обратного рассеяния радиоволн «привязаны» к зонам конвергенции и дивергенцию поверхностного течения.

\item Полученные научные результаты реализованы в виде алгоритмов и элементов программного обеспечения для обработки РСА и оптических изображений и восстановления статистических параметров поверхности океана. А также как элемент разрабатываемой синергетической платформы SYNTool (\url{http://syntool.solab.rshu.ru/}) Лаборатории спутниковой океанографии (ЛСО) РГГМУ.

\end{itemize}

Предлагаемый подход, совместно с существующими РЛ методами наблюдения поверхности Океана, открывает новые возможности для мониторинга океанических явлений из Космоса по их поверхностным проявлениям.

\clearpage