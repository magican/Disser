%%% Макет страницы %%%
%%% в соответствии с положением о представлении обязательного экземпляра диссертаций от 31 августа 1998 года
%Пункт 2.4а)
% текст диссертации должен быть выполнен любым печатным способом на одной стороне листа белой бумаги формата А4 (210x 297 мм) через 1,5-2 межстрочных интервала. Минимально допустимая высота шрифта 1,8 мм. 

\geometry{a4paper,top=2cm,bottom=2.5cm,left=3cm,right=1.5cm}

%%% Font size (12pt,\large) and baselinestretch are fitted %%%
%%% so as to get pages with 30 lines by approx. 60 symbols. %%%
\renewcommand{\baselinestretch}{1.5}

%%% Кодировки и шрифты %%%
%\renewcommand{\rmdefault}{ftm} % Включаем Times New Roman

%%% Выравнивание и переносы %%%
\sloppy					% Избавляемся от переполнений
\clubpenalty=10000		% Запрещаем разрыв страницы после первой строки абзаца
\widowpenalty=10000		% Запрещаем разрыв страницы после последней строки абзаца

%%% Библиография %%%
\bibliographystyle{configs/ugost2008n}
\makeatletter
%\bibliographystyle{configs/utf8gost705u}	% Оформляем библиографию в соответствии с ГОСТ 7.0.5
\renewcommand{\@biblabel}[1]{#1.}	% Заменяем библиографию с квадратных скобок на точку
\let\citep\cite 	%assign the functionality of \cite to \citep when not using natbib author-year citation with style ugost2008n
\makeatother

%%% Изображения %%%
\graphicspath{{images/}} % Пути к изображениям

%%% Цвета гиперссылок %%%
\definecolor{linkcolor}{rgb}{0.9,0.6,0}
\definecolor{citecolor}{rgb}{0,0.6,0}
\definecolor{urlcolor}{rgb}{0,0,1}
\hypersetup{
    colorlinks, linkcolor={linkcolor},
    citecolor={citecolor}, urlcolor={urlcolor}
}

%%% Оглавление %%%
\renewcommand{\cftchapdotsep}{\cftdotsep}

%%%  Расшифровка подписи к рисункам %%%
\newcommand{\floattitle}[1]{%
  \def\floattitletext{#1}% Store float title text
  \itshape
  \centering
  \captiontextfont\strut #1\par\vskip\abovecaptionskip
  }

%use russian alphabet for subfigure counter
\makeatletter
  \def\thesubfigure{\textit{\asbuk{subfigure}}}
  \providecommand\thefigsubsep{,~}
  \def\p@subfigure{\@nameuse{thefigure}\thefigsubsep}
\makeatother

% Нумерованные списки
\renewcommand{\labelenumii}{\arabic{enumi}.\arabic{enumii}.}

% PDF info if using hyperref
\hypersetup{
	pdfinfo={
		Author={Мясоедов Александр Германович},
		Title ={СОЛНЕЧНЫЙ БЛИК КАК «ИНСТРУМЕНТ» ИССЛЕДОВАНИЯ ОКЕАНА ИЗ КОСМОСА},
		Subject={Диссертация},
		Keywords={ocean remote sensing; sun glint; sea surface mean square slope; sea surface slopes probability density function; oil slicks; internal waves; Radar cross sections; Radar imaging; radar signal analysis; Scattering; Sea surface; Surface waves; Synthetic aperture radar; synthetic aperture radar (SAR) imaging}
	}
}